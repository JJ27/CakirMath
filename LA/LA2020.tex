\documentclass{article}
\usepackage[utf8]{inputenc}

\title{Linear Algebra}
\author{Josh Joseph}
\date{Summer 2020}

\addtolength{\oddsidemargin}{-.875in}
\addtolength{\evensidemargin}{-.875in}
\addtolength{\textwidth}{1.75in}
\addtolength{\topmargin}{-.875in}
\addtolength{\textheight}{1.75in}

\usepackage{amsmath}
\usepackage{amssymb}
\usepackage{gensymb}
\usepackage{graphicx}
\usepackage{float}
\usepackage{amsthm}
\usepackage{amsbsy}
\usepackage{etoolbox}
\usepackage{esvect}

\AtBeginEnvironment{gather}{\setcounter{equation}{0}}

\renewcommand{\qedsymbol}{$\blacksquare$}
\let\oldvec\vv
\renewcommand{\vv}[1]{\oldvec{\mathbf{#1}}}
\let\oldhat\hat
\renewcommand{\hat}[1]{\oldhat{\mathbf{#1}}}
\let\vl\langle
\let\vr\rangle
\let\ve\hat
\renewcommand{\ve}[1]{\vl#1\vr}
\let\d\hat
\renewcommand{\d}{\hspace{3pt}\textrm{d}}

\makeatletter
\newcommand*\vdot{\mathpalette\vdot@{.5}}
\newcommand*\vdot@[2]{\mathbin{\vcenter{\hbox{\scalebox{#2}{$\m@th#1\bullet$}}}}}
\makeatother

\begin{document}

\maketitle

\tableofcontents
\newpage
\section{Vectors}
\subsection{The Geometry and Algebra of Vectors}
A \textbf{vector} is a line segment with a direction starting from \textbf{initial point} $A$ to \textbf{terminal point} $B$. For every point $B$ there is a vector $\vv{OB}$ that corresponds to the displacement from $O$ to $B$, where $O$ is the origin. \textit{Refer to MC notes for more on vectors, this is mainly a recap and summarization}.
\subsubsection{Vector Representations}
There are two main ways to represent vectors, the first is a simple \textbf{row vector} while the second is a \textbf{column vector}:
\begin{gather*}
    \vv{v} = \ve{v_x,v_y,v_z...v_n}\\
    \vv{v} = \begin{bmatrix}
    v_x\\
    v_y\\
    v_z\\
    \vdots\\
    v_n
    \end{bmatrix}
\end{gather*}
Two vectors are \textbf{equal} if they have the same components, or if they have the same length(magnitude) and direction. Vectors normally start in the \textbf{standard position}, with the tail at the origin. However, even if a vector is \textit{translated} away, equality does not depend on position.

Vector addition and subtraction are covered in MC notes.
\subsubsection{Linear Combinations}
A vector $\vv{v}$ is a \textbf{linear combination} of vectors $\vv{v_1}, \vv{v_2},\vv{v_3}...\vv{v_k}$ if there are scalars $c_1,c_2,c_3...c_k$ which \textit{combine} to form $\vv{v} = c_1\vv{v_1} + v_2\vv{v_2} + c_3\vv{v_3}...c_k\vv{v_k}$. The scalars are called the \textbf{coefficients} of linear combinations.

If $\vv{v}$ is a linear combination of vectors $\vv{v_1} \textrm{ and } \vv{v_2}$, then the coordinates of $\vv{v}$ with respect to $\vv{v_1},\vv{v_2}$ are $c_1$ and $c_2$.
\subsubsection{Binary Vectors and Modular Arithmetic}
Computers use \textit{binary} to communicate, a language made up of only $0$ and $1$. \textbf{Binary Vectors} are vectors that only have components $0$ or $1$. Using this, the rules of normal math operations must be changed so that whenever a number gets above $1$ it has to be reset down. For example, $3 \implies 1$, and $6 \implies 0$. Basically, any odd number is $1$ and even number is $0$. This binary space is denoted by $\mathbb{Z}_2$. If these vectors all have $n$ components then the space consisting of all of them is $\mathbb{Z}_2^n$, where $n$ is called the \textit{length} of the vector. Be careful not to confuse length in $\mathbb{R}$ with length in $\mathbb{Z}$.

In a similar way, the vectors in $\mathbb{Z}_3$ reset every $3$. So to find a mathematical sum of vectors in this space, add up their components and divide by $3$ and their remainder is the correct value to use. This idea is similar to a clock or a periodic function that repeats.
\subsection{Length and Angle: The Dot Product}
\end{document}
